\documentclass[11pt,preprint]{aastex}
 %\documentclass[12pt]{emulateapj}
\usepackage[margin= 1.0in]{geometry}    % See geometry.pdf to learn the layout options. There are lots.
\geometry{letterpaper} % or letter or a5paper or ... etc
\usepackage{float}
\usepackage{amssymb,amsmath}
\usepackage[]{epsfig,graphicx}
\usepackage{color}
\usepackage{verbatim}
\DeclareGraphicsRule{.tif}{png}{.jpg}{`convert Num1 `dirname Num1`/`basename Num1 .tif`.jpg}
\newcommand{\units}[1]{\ensuremath{\, \mathrm{#1}}}
\usepackage{enumitem}
\usepackage{natbib}
\newcommand{\degree}{\ensuremath{^\circ}}
\usepackage[maxfloats=25]{morefloats}
\usepackage[normalem]{ulem}
\usepackage{hyperref}
\usepackage{ amssymb }
\newcommand{\TRANSPOSE}{\ensuremath{T}}

\bibliographystyle{apalike}


\begin{document}
\title{Cure Violence!}

 \author{Andy Enkebol, Erin Grand, and Mayank Misra}
 \affil{Data Science Institute, Columbia University, New York, NY 10027}
 
\date{\today}             

\begin{abstract}
This is the abstract.
\end{abstract}


\tableofcontents

\section{Introduction}
Cure Violence is a non-profit focused on stopping the spread of violence (e.g., gun violence) in communities (both domestic and international) by using the methods and strategies associated with disease control � detecting and interrupting conflicts, identifying and treating the highest risk individuals, and changing social norms. Specifically, they focus on:
\begin{enumerate}
\item Detecting and interrupting potentially violent conflicts by preventing retaliations and mediating conflicts by being on-the- ground within communities
\item Identifying and treating highest risk through assessing high risk candidates, changing behavior, and providing treatment through 1-on-1 case work
\item Mobilizing the community to change norms by responding to every shooting, organize community events and people, and spreading positive norms
\end{enumerate}





\section{Approach} 
description of data driven process from ideation to formalization of recommendations

\section{Plan / Milesones}
Cost benefit analysis - simple
How many Incidents were avoided, multiplied by the cost saving scalar
Extract data from reports
Cost benefit analysis - complex
Comparison to national averages in Chicago/Baltimore
Exploration of the government data
Lift of incidents avoided
Joining and consolidating internal and external government datasets
Solid path forward for cure violence of where to go next from a cost benefit analysis perspective


%\section{Consolidated} 
%findings and analysis to inform strategic direction
%\section{Operational Recommendations for Cure Violence}
%for scale and prioritization
%\section{Data Strategy recommendations for Cure Violence going forward}
%\section{Lessons learned and follow on work suggestions}


\section{Data Collections and Analysis}
\subsection{Public Reports}
Johns Hopkins University: Baltimore, 2012


Data from the Baltimore Police Department for homicides and nonfatal shootings from January 1, 2003 to December 31, 2010
Surveys of program participants from 2007 - 2010
Average of 30 - 40 participants taking and tracking surveys

Three of the four program sites experienced large, statistically significant, program related reductions in homicides or nonfatal shootings without having a counter-balancing significant increase in one of these outcome measures.


UI Chicago: Chicago, 2009

Northwestern University: Chicago, 2014

Raw crime counts show a 31\% reduction in homicide, a 7\% reduction in total violent crime, and a 19\% reduction in shootings in the targeted districts.
These effects are significantly greater than the effects expected given the declining trends in crime in the city as a whole.
Reduced levels of total violent crime, shootings, and homicides remained constant
The effects of the intervention were immediate
It is not likely that effects were due to increased police activity


Mother Jones: What Does Gun Violence Really Cost?

\subsection{Open Data}
311 and 911 data
Crime reports
Census/population data


\subsection{Proprietary Data}
Case notes (interventions)
Conflict mediation reports
Community Events \& participants
Crime data reported specifically in the bounds of the neighborhood



\newpage
\bibliography{msdref}

\end{document} 
